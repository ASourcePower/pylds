\documentclass{article}

\usepackage[numbers, compress]{natbib}

\usepackage[utf8]{inputenc} % allow utf-8 input
\usepackage[T1]{fontenc}    % use 8-bit T1 fonts
\usepackage{hyperref}       % hyperlinks
\usepackage{url}            % simple URL typesetting
\usepackage{booktabs}       % professional-quality tables
\usepackage{amsfonts}       % blackboard math symbols
\usepackage{nicefrac}       % compact symbols for 1/2, etc.
\usepackage{microtype}      % microtypography

\usepackage{amsthm,amsmath,amssymb}
\usepackage{macros}
\usepackage{subcaption}
\usepackage[textfont=small, labelfont=small]{caption}
\usepackage{graphicx}
\DeclareGraphicsExtensions{.pdf,.png,.jpg,.eps}

\usepackage{algorithm}
\usepackage{algorithmic}

\usepackage[color=yellow]{todonotes}
\usepackage{booktabs}
\usepackage[inline]{enumitem}
\usepackage{verbatim}

\usepackage[margin=1in]{geometry}

\usepackage{setspace}


\title{Information form filtering and smoothing for Gaussian linear dynamical systems}

\author{Scott W. Linderman 
  \and
  Matthew J. Johnson 
}


\begin{document}

\singlespacing
\maketitle
\onehalfspacing

The \emph{information form} of the Gaussian distribution is defined as,
\begin{align}
  p(x \given J, h) 
  &=  \exp \left \{ -\frac{1}{2} x^\trans J x + h^\trans x - \log Z \right \},
\end{align}
where
\begin{align}
  \log Z = \frac{1}{2} h^\trans J^{-1} h - \frac{1}{2}\log |J|.
\end{align}
The standard formulation is recovered by the transformations,  $\Sigma = J^{-1}$, and
$\mu = J^{-1} h$. The advantage of working in the information form is that 
it corresponds to the natural parameterization of the Gaussian distribution, and
mean field variational inference is considerably easier with this form.

In order to perform Kalman filtering and smoothing, we must be able to perform
two operations: \emph{conditioning} and \emph{marginalization}. 

\paragraph{Conditioning}
If,
\begin{align}
  p(x) &= \distNormal(x \given J, h) \\
  p(y \given x) &\propto \distNormal(x \given J_{\sf obs}, h_{\sf obs}) 
\end{align}
then,
\begin{align}
  p(x \given y) &= \distNormal(x \given J + J_{\sf obs}, h + h_{\sf obs}).
\end{align}

\paragraph{Marginalization}
If,
\begin{align}
  \begin{bmatrix} x \\ y  \end{bmatrix}
  &\sim
  \distNormal \left(
  \begin{bmatrix} 
    J_{xx}        & J_{xy} \\
    J_{xy}^\trans & J_{yy} 
  \end{bmatrix},
  \begin{bmatrix} h_{x} \\ h_y \end{bmatrix}
  \right),
\end{align}
then,
\begin{align}
  x &\sim \distNormal(
        J_{xx} - J_{xy}J_{yy}^{-1} J_{xy}^\trans, \;
        h_x - J_{xy} J_{yy}^{-1} h_y)
\end{align}

\section*{Filtering, Sampling, and Smoothing}
By interleaving these two steps we can filter, sample, and smooth the latent states
in a linear dynamical system. Take the model,
\begin{align}
  x_1 &\sim \distNormal(\mu_1, Q_1) \\
  x_{t+1} &\sim \distNormal(A_t x_t + B_t u_t, Q_t) \\
  y_t &\sim \distNormal(C_t x_t + D_t u_t, R_t).
\end{align}
In information form, the initial distribution is,
\begin{align}
  % initi
  x_1 &\sim \distNormal(J=Q_1^{-1}, h = Q_1^{-1} mu_1).
\end{align}
The dynamics are given by,
\begin{align}
  % dynamics
  p(x_{t+1} \given x_t) & \propto 
  \distNormal \left(
  \begin{bmatrix} x_t \\ x_{t+1}  \end{bmatrix}
  \, \bigg| \,
  \begin{bmatrix} 
    J_{11}        & J_{12} \\
    J_{12}^\trans & J_{22} 
  \end{bmatrix},
  \begin{bmatrix} h_{1} \\ h_2 \end{bmatrix}
  \right), 
\end{align}
with,
\begin{align}
  J_{11} &= A_t^\trans Q_t^{-1} A_t, \quad 
  J_{12} = -A_t^\trans Q_t^{-1} \quad
  J_{22} = Q_t^{-1} \quad
  h_1 = -u_t^\trans B_t^\trans Q_t^{-1} A_t \quad
  h_2 = u_t^\trans B_t Q_t^{-1}.
\end{align}
Finally, the observations are given by,
\begin{align}
  p(y_t \given x_t) 
  &\propto \distNormal(x_t \given J_{\sf obs}, h_{\sf obs}) 
\end{align}
with
\begin{align}
  J_{\sf obs} = C_t^\trans R_t^{-1} C_t \quad
  h_{\sf obs} = (y_t - D_t u_t)^\trans R_t^{-1} C_t
\end{align}

\subsection*{Filtering}
We seek the conditional distribution,~$p(x_t \given y_{1:t})$, which 
will be Gaussian. We begin with the initial distribution, 
\begin{align}
  p(x_1) &= \distNormal(x_1 \given J_{1 \given 0}, h_{1 \given 0}).
\end{align}
Assume, inductively, that~$x_t \given y_{1:t-1} \sim \distNormal(J_{t|t-1}, h_{t|t-1})$. Conditioning on the~$t$-th observation yields,
Conditioned on the first observation,
\begin{align}
  p(x_t \given y_{1:t}) &= \distNormal(x_t \given J_{t|t}, h_{t|t}), \\
  J_{t|t} &= J_{t|t-1} + J_{\sf obs} \\
  h_{t|t} &= h_{t|t-1} + h_{\sf obs}.
\end{align}
Then, we predict the next latent state by writing the joint distribution of~$x_t$ and~$x_{t+1}$ and marginalizing out~$x_t$.
\begin{align}
  p(x_{t+1} \given y_{1:t}) &= p(x_t \given y_{1:t}) \, p(x_{t+1} \given x_t) \\
  &= \distNormal(x_t \given J_{t+1|t}, h_{t+1|t}) \\
  J_{t+1|t} &= J_{22} - J_{21} (J_{t|t} + J_{11})^{-1}J_{21}^\trans \\
  h_{t+1|t} &= h_2 - J_{21} (J_{t|t} + J_{11})^{-1} (h_{t|t} + h_1)
\end{align}.
This completes one iteration and provides the input to the next. To start the recursion, we initialize,
\begin{align}
  J_{1|0} = \Sigma_{\sf init}^{-1}, \quad
  h_{1|0} = \Sigma_{\sf init}^{-1} \, \mu_{\sf init}.
\end{align}

\subsection*{Backward Sampling}
Having computed~$J_{t|t}$ and~$h_{t|t}$, we the proceed backward in time to draw a joint sample of the latent states. 
Given~$J_{t|t}$,~$h_{t|t}$, and~$x_{t+1}$, we have,
\begin{align}
  p(x_{t} \given y_{1:t}, x_{t+1}) &\propto p(x_t \given y_{1:T}) \, p(x_{t+1} \given x_t) \\
  &\propto \distNormal(x_t \given J_{t|t}, h_{t|t})  \;
    \distNormal(x_t \given J_{11}, \, h_1 - x_{t+1}^\trans J_{21} ) \\
  &\propto     \distNormal(x_t \given J_{t|t} + J_{11}, \; h_{t|t} + h_1 - x_{t+1}^\trans J_{21} )
\end{align}
We sample~$x_t$ from this conditional, then use it to sample~$x_{t-1}$, and repeat until we reach~$x_1$.

\subsection*{Rauch-Tung-Striebel Smoothing}
Next we seek the conditional distribution given all the data,~$p(x_t \given y_{1:T})$. 
This will again be Gaussian, and we will call its parameters~$J_{t|T}$ and~$h_{t|T}$. 
Assume, inductively, that we have computed~$J_{t+1|T}$ and~$h_{t+1|T}$. We show how to 
compute the parameters for time~$t$.

From the Markov properties of the model and the conditional distribution derived above, we have,
\begin{align}
  p(x_t \given x_{t+1}, y_{1:T}) 
  &= \distNormal(x_t \given J_{t|t} + J_{11}, \; h_{t|t} + h_1 - J_{12} x_{t+1}).
\end{align}
Expanding, taking care to note that~$x_{t+1}$ appears in the normalizing constant, yields,
\begin{multline}
  p(x_t \given x_{t+1}, y_{1:T}) 
  = \exp \bigg \{-\frac{1}{2} x_t^\trans (J_{t|t} + J_{11}) x_t + (h_{t|t} + h_1)^\trans x_t - x_{t+1}^\trans J_{12} x_t 
  \\ 
-\frac{1}{2} x_{t+1}^\trans J_{12}^\trans (J_{t|t} + J_{11})^{-1} J_{12} x_{t+1} 
+(h_{t|t} + h_1)^\trans (J_{t|t} + J_{11})^{-1} J_{12} x_{t+1} \\
- \frac{1}{2} (h_{t|t} + h_1)^\trans (J_{t|t} + J_{11})^{-1} (h_{t|t} + h_1) 
    \bigg \}
\end{multline}

Now consider the joint distribution of~$x_t$ and~$x_{t+1}$ given all the data,
\begin{align}
  p(x_t, x_{t+1} \given y_{1:T}) 
  &= p(x_t \given x_{t+1}, y_{1:T}) p(x_{t+1} \given y_{1:T}) \\
  &\propto \distNormal \left(
  \begin{bmatrix} x_t \\ x_{t+1}  \end{bmatrix}
  \, \bigg| \,
  \begin{bmatrix} 
    \widetilde{J}_{11}        & \widetilde{J}_{12} \\
    \widetilde{J}_{12}^\trans & \widetilde{J}_{22} 
  \end{bmatrix},
  \begin{bmatrix} 
    \widetilde{h}_1 \\ 
    \widetilde{h}_2 
  \end{bmatrix}
  \right), 
\end{align}
with,
\begin{align}
  \widetilde{J}_{11} &= J_{t|t} + J_{11} \\ 
  \widetilde{J}_{12} &= J_{12} \\
  \widetilde{J}_{22} &= J_{t+1|T} + J_{12}^\trans (J_{t|t} + J_{11})^{-1} J_{12} \\
  \widetilde{h}_1 &= h_{t|t} + h_1 \\
  \widetilde{h}_2 &= h_{t+1|T} + (h_{t|t} + h_1)^\trans (J_{t|t} + J_{11})^{-1} J_{12}.
\end{align}
Recall that,
\begin{align}
  J_{t+1|t} &= J_{22} - J_{21} (J_{t|t} + J_{11})^{-1}J_{21}^\trans \\
  h_{t+1|t} &= h_2 - J_{21} (J_{t|t} + J_{11})^{-1} (h_{t|t} + h_1).
\end{align}
Thus, 
\begin{align}
  \widetilde{J}_{22} &= J_{t+1|T} - J_{t+1|t} + J_{22} \\
  \widetilde{h}_{2} &= h_{t+1|T} - h_{t+1|t} + h_2.
\end{align}


Finally, marginalize,
\begin{align}
  p(x_t \given y_{1:T}) 
  &= \distNormal(x_t \given 
    \widetilde{J}_{11} - \widetilde{J}_{12} \widetilde{J}_{22}^{-1} \widetilde{J}_{12}^\trans, \;
    \widetilde{h}_{1} - \widetilde{J}_{12} \widetilde{J}_{22}^{-1} \widetilde{h}_2) \\
  &= \distNormal(x_t \given J_{t|T}, h_{t|T}).
\end{align}
Substituting the simplified forms above yields,
\begin{align}
  J_{t|T} &= J_{t|t} + J_{11} - J_{12} (J_{t+1|T} - J_{t+1|t} + J_{22})^{-1} J_{12}^\trans \\
  h_{t|T} &= h_{t|t} + h_1 - J_{12} (J_{t+1|T} - J_{t+1|t} + J_{22})^{-1} (h_{t+1|T} - h_{t+1|t} +h_2).
\end{align}

\end{document}